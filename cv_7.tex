%%%%%%%%%%%%%%%%%%%%%%%%%%%%%%%%%%%%%%%%%
% "ModernCV" CV and Cover Letter
% LaTeX Template
% Version 1.3 (29/10/16)
%
% This template has been downloaded from:
% http://www.LaTeXTemplates.com
%
% Original author:
% Xavier Danaux (xdanaux@gmail.com) with modifications by:
% Vel (vel@latextemplates.com)
%
% License:
% CC BY-NC-SA 3.0 (http://creativecommons.org/licenses/by-nc-sa/3.0/)
%
% Important note:
% This template requires the moderncv.cls and .sty files to be in the same 
% directory as this .tex file. These files provide the resume style and themes 
% used for structuring the document.
%
%%%%%%%%%%%%%%%%%%%%%%%%%%%%%%%%%%%%%%%%%

%----------------------------------------------------------------------------------------
%	PACKAGES AND OTHER DOCUMENT CONFIGURATIONS
%----------------------------------------------------------------------------------------

\documentclass[11pt,a4paper,sans]{moderncv} % Font sizes: 10, 11, or 12; paper sizes: a4paper, letterpaper, a5paper, legalpaper, executivepaper or landscape; font families: sans or roman

\moderncvstyle{casual} % CV theme - options include: 'casual' (default), 'classic', 'oldstyle' and 'banking'
\moderncvcolor{blue} % CV color - options include: 'blue' (default), 'orange', 'green', 'red', 'purple', 'grey' and 'black'

\usepackage{lipsum} % Used for inserting dummy 'Lorem ipsum' text into the template

\usepackage[scale=0.75]{geometry} % Reduce document margins
%\setlength{\hintscolumnwidth}{3cm} % Uncomment to change the width of the dates column
%\setlength{\makecvtitlenamewidth}{10cm} % For the 'classic' style, uncomment to adjust the width of the space allocated to your name

%----------------------------------------------------------------------------------------
%	NAME AND CONTACT INFORMATION SECTION
%----------------------------------------------------------------------------------------

\firstname{Dario} % Your first name
\familyname{Gangi} % Your last name

% All information in this block is optional, comment out any lines you don't need
\title{Curriculum Vitae}
\address{Piazza Giacomo Matteotti, 40}{57126 - Livorno (ITALIA)}
\mobile{(+39) 333 6567094}
%\phone{(000) 111 1112}
%\fax{(000) 111 1113}
\email{dariogangi@hotmail.com}
%\homepage{staff.org.edu/~jsmith}{staff.org.edu/$\sim$jsmith} % The first argument is the url for the clickable link, the second argument is the url displayed in the template - this allows special characters to be displayed such as the tilde in this example
%\extrainfo{additional information}
\photo[70pt][0.4pt]{pictures/foto} % The first bracket is the picture height, the second is the thickness of the frame around the picture (0pt for no frame)
%\quote{"A witty and playful quotation" - John Smith}

%----------------------------------------------------------------------------------------

\begin{document}

%----------------------------------------------------------------------------------------
%	CURRICULUM VITAE
%----------------------------------------------------------------------------------------

\makecvtitle % Print the CV title

%----------------------------------------------------------------------------------------
%	EDUCATION SECTION
%----------------------------------------------------------------------------------------

\section{Istruzione}
\cventry{2001--2006}{Diploma di Ragioniere Programmatore}{ITC Paolo Dagomari}{Via di Reggiana, 86, 59100 - Prato (Italia)}{}{\url{http://www.itesdagomari.it/ }}
\cventry{2007--2013}{Laurea Triennale in Informatica}{Universit\`{a} degli Studi di Roma "La Sapienza"}{Via Salaria, 113, 00198 - Roma (Italia)}{}{\url{http://www.studiareinformatica.uniroma1.it}}
\cventry{2007--2008}{Certificato in Videogame Programming}{Accademia Italiana Videogiochi}{Viale Ippocrate, 73, 00161 - Roma (Italia)}{}{\url{http://www.aiv01.it}}
\cventry{2013--2016}{Laurea Magistrale in Informatica}{Universit\`{a} degli Studi di Roma "La Sapienza"}{Via Salaria, 113, 00198 - Roma (Italia)}{}{\url{http://www.studiareinformatica.uniroma1.it}}

\section{Tesi di Laurea Magistrale}

\cvitem{Titolo}{\emph{Un Sistema per l'Analisi dei Cambiamenti in Immagini Acquisite da UAV per la Sorveglianza Attiva}}
\cvitem{Relatore}{Professore Luigi Cinque}
\cvitem{Descrizione}{Progettazione e sviluppo di un sistema di sorveglianza attiva che ha l'obiettivo di rilevare dei cambiamenti, in particolare in ambienti outdoor non urbani. Tale sistema dovr\`{a} analizzare le immagini aeree acqusite da un drone UAV che monitora una specifica zona.}

%----------------------------------------------------------------------------------------
%	WORK EXPERIENCE SECTION
%----------------------------------------------------------------------------------------

\section{Esperienze}

\subsection{Professionali}

\cventry{11/2008--01/2011}{Programmatore}{Elemental srl}{Viale Ippocrate, 73, 00161 - Roma (Italia)}{}
{
	\textbf{Attivit\`{a} o settore:} Game Development
	\newline{}
	\url{http://www.elementaldesign.it/}
	\newline{}
	\url{http://www.aiv01.it/} 
	\newline{}
	\begin{itemize}
		\item Programmazione C++
		\item Utilizzo delle librerie grafiche DirectX 9.0c
		\item Programmazione C\#
		\item Utilizzo delle librerie grafiche XNA
		\begin{itemize}
			\item Gestione 3D delle telecamere
			\item Implementazione dell'algoritmo Convex Hull per la creazione di mesh fisiche
			\item Gestione della GUI
			\item Implementazione di collisione con un tronco di piramide per il Frustum Culling
			\item Implementazione di A* per l'intelligenza artificiale dei nemici
		\end{itemize}
		\item Programmazione di gameplay del gioco Evil Heroes
		\begin{itemize}
			\item Spostamento dei nemici verso il personaggio principale
			\item Implementazione della barra vitale del personaggio principale
		\end{itemize}
	\end{itemize}
}

\subsection{Varie}

\cventry{11/2012--04/2013}{Programmatore}{Centro di Ricerca C.A.T.T.I.D. Universit\`{a} degli Studi di Roma "La Sapienza"}{Roma}{}
{
	\begin{itemize}
		\item Programmazione di un'applicazione per Windows Phone 7
		\item Utilizzo di C\# e Windows Phone SDK 7.1
	\end{itemize}
}



\cventry{05/2014--06/2014}{Programmatore}{Universit\`{a} degli Studi di Roma "La Sapienza"}{Roma}{}
{
	\url{http://gamificationlab.uniroma1.it/laboratorio/progetto-gamificationlab-2014}
	\begin{itemize}
		\item Programmazione di un piccolo gioco sulla raccolta differenziata
		\item Utilizzo di C\# Unity 4
		\item Presentazione alla fiera Maker Faire di Roma
		\newline{} \url{http://gamificationlab.uniroma1.it/archivionotizie/il-prototipo-del-gioco-differenziati-selezionato-alla-maker-faire}
	\end{itemize}
}
\subsection{Miscellaneous}

\cventry{2010--2011}{}{}{}{}{Spent some time finding myself. This was a courageous endeavour that didn't have a job title. It was quite important to my overall development though so I'm adding it to my CV. Also it explains the gap in my otherwise stellar CV.}

\cventry{2009--2010}{Computer Repair Specialist}{Buy More}{Burbank}{}{Worked in the Nerd Herd and helped to solve computer problems. Allowed me to become expert in all forms of martial arts and weaponry.}

%----------------------------------------------------------------------------------------
%	AWARDS SECTION
%----------------------------------------------------------------------------------------

\section{Awards}

\cvitem{2011}{School of Business Postgraduate Scholarship}
\cvitem{2010}{Top Achiever Award -- Commerce}

%----------------------------------------------------------------------------------------
%	COMPUTER SKILLS SECTION
%----------------------------------------------------------------------------------------

\section{Computer skills}

\cvitem{Basic}{\textsc{java}, Adobe Illustrator}
\cvitem{Intermediate}{\textsc{python}, \textsc{html}, \LaTeX, OpenOffice, Linux, Microsoft Windows}
\cvitem{Advanced}{Computer Hardware and Support}

%----------------------------------------------------------------------------------------
%	COMMUNICATION SKILLS SECTION
%----------------------------------------------------------------------------------------

\section{Communication Skills}

\cvitem{2010}{Oral Presentation at the California Business Conference}
\cvitem{2009}{Poster at the Annual Business Conference in Oregon}

%----------------------------------------------------------------------------------------
%	LANGUAGES SECTION
%----------------------------------------------------------------------------------------

\section{Languages}

\cvitemwithcomment{English}{Mothertongue}{}
\cvitemwithcomment{Spanish}{Intermediate}{Conversationally fluent}
\cvitemwithcomment{Dutch}{Basic}{Basic words and phrases only}

%----------------------------------------------------------------------------------------
%	INTERESTS SECTION
%----------------------------------------------------------------------------------------

\section{Interests}

\renewcommand{\listitemsymbol}{-~} % Changes the symbol used for lists

\cvlistdoubleitem{Piano}{Chess}
\cvlistdoubleitem{Cooking}{Dancing}
\cvlistitem{Running}

%----------------------------------------------------------------------------------------

\end{document}