%%%%%%%%%%%%%%%%%%%%%%%%%%%%%%%%%%%%%%%%%
% "ModernCV" CV and Cover Letter
% LaTeX Template
% Version 1.3 (29/10/16)
%
% This template has been downloaded from:
% http://www.LaTeXTemplates.com
%
% Original author:
% Xavier Danaux (xdanaux@gmail.com) with modifications by:
% Vel (vel@latextemplates.com)
%
% License:
% CC BY-NC-SA 3.0 (http://creativecommons.org/licenses/by-nc-sa/3.0/)
%
% Important note:
% This template requires the moderncv.cls and .sty files to be in the same 
% directory as this .tex file. These files provide the resume style and themes 
% used for structuring the document.
%
%%%%%%%%%%%%%%%%%%%%%%%%%%%%%%%%%%%%%%%%%

%----------------------------------------------------------------------------------------
%	PACKAGES AND OTHER DOCUMENT CONFIGURATIONS
%----------------------------------------------------------------------------------------

\documentclass[11pt,a4paper,sans]{moderncv} % Font sizes: 10, 11, or 12; paper sizes: a4paper, letterpaper, a5paper, legalpaper, executivepaper or landscape; font families: sans or roman

\moderncvstyle{casual} % CV theme - options include: 'casual' (default), 'classic', 'oldstyle' and 'banking'
\moderncvcolor{blue} % CV color - options include: 'blue' (default), 'orange', 'green', 'red', 'purple', 'grey' and 'black'

\usepackage{lipsum} % Used for inserting dummy 'Lorem ipsum' text into the template

\usepackage[scale=0.75]{geometry} % Reduce document margins
%\setlength{\hintscolumnwidth}{3cm} % Uncomment to change the width of the dates column
%\setlength{\makecvtitlenamewidth}{10cm} % For the 'classic' style, uncomment to adjust the width of the space allocated to your name

%----------------------------------------------------------------------------------------
%	NAME AND CONTACT INFORMATION SECTION
%----------------------------------------------------------------------------------------

\firstname{Dario} % Your first name
\familyname{Gangi} % Your last name

% All information in this block is optional, comment out any lines you don't need
\title{Curriculum Vitae}
\address{Piazza Giacomo Matteotti, 40}{57126 - Livorno (Italia)}
\mobile{(+39) 333 6567094}
%\phone{(000) 111 1112}
%\fax{(000) 111 1113}
\email{dariogangi@hotmail.com}
%\homepage{staff.org.edu/~jsmith}{staff.org.edu/$\sim$jsmith} % The first argument is the url for the clickable link, the second argument is the url displayed in the template - this allows special characters to be displayed such as the tilde in this example
%\extrainfo{additional information}
\photo[70pt][0.4pt]{pictures/foto} % The first bracket is the picture height, the second is the thickness of the frame around the picture (0pt for no frame)
%\quote{"A witty and playful quotation" - John Smith}

%----------------------------------------------------------------------------------------

\begin{document}

%----------------------------------------------------------------------------------------
%	CURRICULUM VITAE
%----------------------------------------------------------------------------------------

\makecvtitle % Print the CV title

%----------------------------------------------------------------------------------------
%	EDUCATION SECTION
%----------------------------------------------------------------------------------------

\section{Istruzione}
\cventry{2001--2006}{Diploma di Ragioniere Programmatore}{ITC Paolo Dagomari}{Via di Reggiana, 86, 59100 - Prato (Italia)}{}{\url{http://www.itesdagomari.it/ }}


\cventry{2007--2013}{Laurea Triennale in Informatica}{Universit\`{a} degli Studi di Roma "La Sapienza"}{Via Salaria, 113, 00198 - Roma (Italia)}
{
	\newline{}
	\textit{\textbf{Votazione conseguita:} 103/110}
}
{
	\url{http://www.studiareinformatica.uniroma1.it}
	\begin{itemize}
		\item Conoscenze matematiche avanzate
		\item Conoscenze di programmazione in C e Java
		\item Conoscenze di linguaggi funzionali
		\item Conoscenze base sull'architettura di rete, modelli ISO-OSI e TCP/IP
		\item Conoscenze a basso livello nei sistemi operativi UNIX
		\item Conoscenze dell'architettura di un elaboratore
		\item Conoscenze riguardo la progettazione e l'analisi di un software
		\item Conoscenze teoriche e pratiche riguardo le basi di dati
		\item Progettazione e analisi di algoritmi
	\end{itemize}
}


\cventry{2007--2008}{Certificato in Videogame Programming}{Accademia Italiana Videogiochi}{Viale Ippocrate, 73, 00161 - Roma (Italia)}{}
{
	\url{http://www.aiv01.it}
	\begin{itemize}
		\item Programmazione C\textbackslash C++
		\item Programmazione Orientata ad Oggetti
		\item Utilizzo di Microsoft Visual Studio 2007/2010
		\item Utilizzo di sistemi di software di controllo versione CVS, SVN e Microsoft Team Foundation Version Control
		\item Programmazione con DirectX 9.0c
		\item Programmazione di pixel shader e vertex shader con linguaggio HLSL
	\end{itemize}
}


\cventry{2013--2016}{Laurea Magistrale in Informatica}{Universit\`{a} degli Studi di Roma "La Sapienza"}{Via Salaria, 113, 00198 - Roma (Italia)}
{
	\newline{}
	\textit{\textbf{Votazione conseguita:} 110/110 e lode}
}
{
	\url{http://www.studiareinformatica.uniroma1.it}
	\begin{itemize}
		\item Studio e programmazione in ambito della grafica computazionale (algoritmo di ray tracing, OpenGL, GLSL)
		\item Elaborazione delle immagini nell'ambito della visione artificiale e dei sistemi biometrici (utilizzo di OpenCV)
		\item Studio nell'ambito dell'Information Retrivial (utilizzo delle Twitter API, tramite Twitter4j)
		\item Studio di Big Data e utilizzo di Hadoop tramite Amazon Web Service
		\item Studio nell'ambito dell'informatica teorica (calcolabilit\`{a}, complessit\`{a}, lambda calcolo)
		\item Studio approfondito sulla progettazione, gestione delle reti e cloud computing
	\end{itemize}
}

\section{Tesi di Laurea Magistrale}

\cvitem{Titolo}{\emph{Un Sistema per l'Analisi dei Cambiamenti in Immagini Acquisite da UAV per la Sorveglianza Attiva}}
\cvitem{Relatore}{Professore Luigi Cinque}
\cvitem{Descrizione}{Progettazione e sviluppo di un sistema di sorveglianza attiva che ha l'obiettivo di rilevare dei cambiamenti, in particolare in ambienti outdoor non urbani. Tale sistema dovr\`{a} analizzare le immagini aeree acqusite da un drone UAV che monitora una specifica zona. Il sistema è stato sviluppato in linguaggio C++ con l'ausilio della libreria OpenCV.}

%----------------------------------------------------------------------------------------
%	WORK EXPERIENCE SECTION
%----------------------------------------------------------------------------------------

\section{Esperienze}

\subsection{Professionali}

\cventry{11/2008--01/2011}{Programmatore}{Elemental srl}{Viale Ippocrate, 73, 00161 - Roma (Italia)}{}
{
	\textbf{Attivit\`{a} o settore:} Game Development
	\newline{}
	\url{http://www.elementaldesign.it/}
	\newline{}
	\url{http://www.aiv01.it/} 
	\begin{itemize}
		\item Programmazione C++
		\item Utilizzo delle librerie grafiche DirectX 9.0c
		\item Programmazione C\#
		\item Utilizzo delle librerie grafiche XNA
		\begin{itemize}
			\item Gestione 3D delle telecamere
			\item Implementazione dell'algoritmo Convex Hull per la creazione di mesh fisiche
			\item Gestione della GUI
			\item Implementazione di collisione con un tronco di piramide per il Frustum Culling
			\item Implementazione di A* per l'intelligenza artificiale dei nemici
		\end{itemize}
		\item Programmazione di gameplay del gioco Evil Heroes
		\begin{itemize}
			\item Spostamento dei nemici verso il personaggio principale
			\item Implementazione della barra vitale del personaggio principale
		\end{itemize}
	\end{itemize}
}

\subsection{Varie}

\cventry{11/2012--04/2013}{Programmatore}{Centro di Ricerca C.A.T.T.I.D. Universit\`{a} degli Studi di Roma "La Sapienza"}{Roma}{}
{
	\begin{itemize}
		\item Tirocinio per Corso di Laurea Triennale in Informatica
		\item Sviluppo di un'applicazione per Windows Phone 7
		\item Utilizzo di C\# e Windows Phone SDK 7.1
	\end{itemize}
}



\cventry{04/2014--06/2014}{Programmatore}{Universit\`{a} degli Studi di Roma "La Sapienza"}{Roma}{}
{
	\url{http://www.electricitylab.net/blog/9-news/46-differenziati-un-tuo-gesto-migliorera-il-mondo.html}
	\newline{}
	\url{http://gamificationlab.uniroma1.it/laboratorio/progetto-gamificationlab-2014}
	\begin{itemize}
		\item Sviluppo di un piccolo gioco sulla raccolta differenziata
		\item Utilizzo di C\# Unity 4, Arduino, Kinect for Windows SDK
		\item Presentazione alla fiera Maker Faire di Roma
		\newline{} \url{http://gamificationlab.uniroma1.it/archivionotizie/il-prototipo-del-gioco-differenziati-selezionato-alla-maker-faire}
	\end{itemize}
}



\cventry{05/2014--09/2014}{Programmatore}{Universit\`{a} degli Studi di Roma "La Sapienza"}{Roma}{}
{
	\begin{itemize}
		\item Sviluppo di un'applicazione Android per il Museo di Antropologia "Giuseppe Sergi" di Roma di un piccolo gioco sulla raccolta differenziata
		\item Utilizzo del motore grafico Unity 4
		\item Implementazione di un'activity per la visualizzazione di modelli 3D di crani esposti al museo
		\item Pubblicazione su Google Play
		\newline{}
		\url{https://play.google.com/store/apps/details?id=sapienza.informatica.man}
	\end{itemize}
}
%----------------------------------------------------------------------------------------
%	AWARDS SECTION
%----------------------------------------------------------------------------------------

\section{Awards}

\cvitem{2011}{School of Business Postgraduate Scholarship}
\cvitem{2010}{Top Achiever Award -- Commerce}

%----------------------------------------------------------------------------------------
%	COMPUTER SKILLS SECTION
%----------------------------------------------------------------------------------------

\section{Computer skills}

\cvitem{Basic}{\textsc{java}, Adobe Illustrator}
\cvitem{Intermediate}{\textsc{python}, \textsc{html}, \LaTeX, OpenOffice, Linux, Microsoft Windows}
\cvitem{Advanced}{Computer Hardware and Support}

%----------------------------------------------------------------------------------------
%	COMMUNICATION SKILLS SECTION
%----------------------------------------------------------------------------------------

\section{Communication Skills}

\cvitem{2010}{Oral Presentation at the California Business Conference}
\cvitem{2009}{Poster at the Annual Business Conference in Oregon}

%----------------------------------------------------------------------------------------
%	LANGUAGES SECTION
%----------------------------------------------------------------------------------------

\section{Languages}

\cvitemwithcomment{English}{Mothertongue}{}
\cvitemwithcomment{Spanish}{Intermediate}{Conversationally fluent}
\cvitemwithcomment{Dutch}{Basic}{Basic words and phrases only}

%----------------------------------------------------------------------------------------
%	INTERESTS SECTION
%----------------------------------------------------------------------------------------

\section{Interests}

\renewcommand{\listitemsymbol}{-~} % Changes the symbol used for lists

\cvlistdoubleitem{Piano}{Chess}
\cvlistdoubleitem{Cooking}{Dancing}
\cvlistitem{Running}

%----------------------------------------------------------------------------------------

\end{document}